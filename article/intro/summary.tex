\pagestyle{plain}
\begin{center}
{\LARGE Abstract}\\[1cm]
\end{center}

Microservice architecture is a widely adopted design approach in modern software development that emphasizes the development of specialized services with clearly defined capabilities and functions. This design approach has gained popularity with businesses attempting to increase responsiveness and reduce and simplify application development cycles. The improvements are realized by the integration of software engineering practices and IT operations that facilitates continuous integration, testing, and delivery, with minimal, if any, downtime. Microservices address the limitations of monolithic architectures, traditionally used in software development, by promoting modularity and independence of services. Every microservice is a separate process, and thus, development, deployment, and scaling can be achieved independently. The services may be developed with various programming languages and data storage mechanisms, thus enabling more flexibility and easy optimization. Although this model drastically improves scalability and maintainability, it requires effective orchestration and communication among the services. In today's world, vast volumes of data are being produced from various sources, such as Internet of Things (IoT) systems. However, this data is valuable only when properly processed, analysed and presented. To achieve this, technologies such as data caching, visualization, processing, and analytics are integrated into microservice-based systems. The use of these methods can enhance the amount of real-time data available, enabling and enhancing data driven decision-making in multiple sectors, such as energy management, smart cities, and industrial automation.